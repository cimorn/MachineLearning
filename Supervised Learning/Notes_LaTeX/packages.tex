\usepackage{xcolor}
%\definecolor{Sumire}{RGB}{102,50,124}
%\definecolor{Shion}{RGB}{143,119,181}
\definecolor{ecolor}{RGB}{1,126,218}

\usepackage{tikz-cd}
\usepackage{titlesec}
\usepackage{titleps}
\usepackage{tikz}
\usetikzlibrary{arrows.meta}


\newcommand{\chapnumfont}{%
	\fontsize{100}{100}\usefont{T1}{ptm}{b}{n}%
}

\colorlet{chapbgcolor}{gray!75}
\colorlet{chapnumcolor}{black!60}


\newcommand{\chaptitlenumbered}[1]{%
	\begin{tikzpicture}
		\fill[chapbgcolor!70,rounded corners=0pt] (0,2.3) rectangle (\linewidth,0);
		\node[
		align=right,
		anchor=south east,
		inner sep=8pt,
		font=\LARGE\normalfont\bfseries
		] at (0.987\linewidth,0) {\strut#1};
		\node[
		align=right,
		font=\fontsize{60}{62}\usefont{OT1}{ptm}{b}{it},
		text=chapnumcolor
		] at (0.975\linewidth,2.3) {\thechapter};
	\end{tikzpicture}%
}

\newcommand{\chaptitleunnumbered}[1]{%
	\begin{tikzpicture}
		\fill[chapbgcolor!70,rounded corners=0pt] (0,2.3) rectangle (\linewidth,0);
		\node[
		align=right,
		anchor=south east,
		inner sep=8pt,
		font=\huge\normalfont\bfseries
		] at (0.987\linewidth,0) {\strut#1};
	\end{tikzpicture}%
}

\usepackage{titlesec}

\titleformat{name=\chapter}[display]
{\normalfont\huge\bfseries\sffamily}
{}
{40pt}
{\chaptitlenumbered}
\titleformat{name=\chapter,numberless}[display]
{\normalfont\huge\bfseries\sffamily}
{}
{25pt}
{\chaptitleunnumbered}
\titlespacing*{\chapter}
{0pt}
{-126pt}
{33pt}

\titleformat{\section}
{\normalfont\Large\bfseries\color{red}}
{{\mdseries\S}\ \thesection}{.5em}{}

\titleformat{\subsection}
{\normalfont\large\bfseries\color{red}}
{{\mdseries\S}\ \thesubsection}{.5em}{}

\setlength\headheight{15pt}

\usepackage{dsfont}

\renewcommand{\baselinestretch}{1.25}

\renewcommand\sectionmark[1]{%
	\markright{\thesection\quad #1}}

% \usepackage{indentfirst}
\setlength{\parindent}{0pt}
% 无首航缩进

\usepackage{parskip}
\setlength{\parskip}{0.5\baselineskip} % 设置段落间距为两倍的基线间距,即空两行

\usepackage{xeCJK}
%\setCJKmainfont{PingFangSC-Regular}


%设定字体

%% \setCJKmainfont[ItalicFont=STKaiti]{Source Han Serif CN}
%% 为思源宋体

%\setCJKmainfont[BoldFont=黑体, ItalicFont=STKaiti]{Source Han Serif CN}
%% 为中易黑体

%% \setCJKsansfont{Source Han Sans CN}
%% \setCJKmonofont{Source Han Sans CN}
%% \setCJKfamilyfont{boldsong}{Source Han Serif CN Heavy}
%\normalspacedchars{*}
%数学式
\usepackage{mathtools,extarrows,amsfonts,amssymb,bm,mathrsfs} %数学式宏包, 更多箭头, 黑板体等数学字体,leqslant等符号
\usepackage{amsthm} %定理环境

\allowdisplaybreaks
\usepackage{ulem}
% 下划线

\DeclareMathOperator{\rad}{rad}
\DeclareMathOperator{\diam}{diam}
\DeclareMathOperator{\fin}{fin}
\DeclareMathOperator{\esssup}{ess,sup}
\DeclareMathOperator{\conv}{Conv}
\DeclareMathOperator{\Span}{span} %% 因为\span已经在宏中定义, 这里使用大写的\Span来表示线性张成
\DeclareMathOperator{\cont}{Cont} %% 表示函数的连续点
\DeclareMathOperator{\diag}{diag}
\DeclareMathOperator{\codim}{codim}
\DeclareMathOperator{\convba}{Convba}


\newcommand{\me}{\ensuremath{\mathrm{e}}}
\newcommand{\imag}{\mathrm{i}}
\newcommand{\Star}[1]{#1^{*}}
\newcommand{\1}{\mathds{1}}
\newcommand{\supp}{\mathrm{supp},}

%% \C已被定义, 重定义在交叉引用部分
\newcommand{\R}{\ensuremath{\mathbb{R}}}
\newcommand{\J}{\ensuremath{\mathbb{J}}}
\newcommand{\Q}{\ensuremath{\mathbb{Q}}}
\newcommand{\Z}{\ensuremath{\mathbb{Z}}}
\newcommand{\N}{\ensuremath{\mathbb{N}}}
\newcommand{\K}{\ensuremath{\mathbb{K}}}
\newcommand{\Zo}{\ensuremath{\mathbb{Z}_{\geqslant 0}}} % 非负整数集
\newcommand{\Zi}{\ensuremath{\mathbb{Z}_{\geqslant 1}}} % 正整数集
\newcommand{\id}{\mathrm{id}}
\newcommand{\im}{\mathrm{im},}                         % 映射的像

\newcommand{\CA}{\mathcal{A}}
\newcommand{\CB}{\mathcal{B}}
\newcommand{\CF}{\mathcal{F}}
\newcommand{\CG}{\mathcal{G}}
\newcommand{\CH}{\mathcal{H}}
\newcommand{\CK}{\mathcal{K}}
\newcommand{\CL}{\mathcal{L}}
\newcommand{\CN}{\mathcal{N}}

\newcommand{\FB}{\mathfrak{B}}

\renewcommand{\Re}{\mathrm{Re,}}
\renewcommand{\Im}{\mathrm{Im,}}
\newcommand{\sgn}{\mathrm{sgn},}
\newcommand{\diff}{,\mathrm{d}}

\newcommand{\Fs}{\ensuremath{\CF_{\sigma}}}
\newcommand{\Gd}{\ensuremath{\CG_{\delta}}}
\newcommand{\Fr}{\ensuremath{\CF_{r}}}


%\usepackage{tasks}
%\NewTasks[counter-format=(tsk[1]), item-indent=2em, label-offset=1em, label-align=right]{lpbn}
%\NewTasks[counter-format=(tsk[a]), item-indent=2em, label-offset=1em]{alpbn}
%\NewTasks[counter-format=tsk[A].]{xrze}[*]

%版式
\usepackage[a4paper,left=2.5cm,right=2.5cm,top=2.5cm,bottom=2cm]{geometry} %边距
%\usepackage[a4paper,left=1.8cm,right=3.2cm,top=2.5cm,bottom=2cm]{geometry} %当打印时使用此选项
\setlength{\headheight}{13pt}
\usepackage{pifont}		% 带圈数字
\usepackage{fancyhdr} 	% 页眉页脚
\pagestyle{fancy}
\fancyhf{}
\fancyhead[OL]{\Roman \nouppercase\rightmark}
\fancyhead[ER]{\Roman \nouppercase\leftmark}
\fancyhead[OR,EL]{\thepage}
\fancyfoot[C]{}
\usepackage{tocbibind}
\usepackage{imakeidx}

%辅助
\usepackage{array,diagbox,booktabs,tabularx} 
%数组环境, 表格中可以添加对角线, 可以调整表格中线的宽度, 可以控制表格宽度并使其自动换行
\usepackage{enumitem} % 继承并扩展了enumerate宏包的功能
% 对itemize环境进行设置
\setlist[itemize]{
	labelindent=\parindent, % 项目符号缩进一个缩进距离
	labelsep=0.5em, % 内容和项目符号之间的间距设为1em,可按需调整
	leftmargin=2em, 
	noitemsep, % 去除列表项之间的额外垂直间距
	itemindent=0pt % 列表项内容不额外缩进
}
\setlength{\topsep}{0ex} % 列表到上下文的垂直距离

%交叉引用
\usepackage{nameref}
\usepackage{prettyref}
\usepackage[colorlinks, linkcolor=red,citecolor=red,urlcolor=black]{hyperref}

\usepackage{graphicx}
\newcommand{\C}{\ensuremath{\mathbb{C}}} 
%浮动体
\usepackage{caption} %使用浮动体标题
\usepackage{subfig} %子浮动体

% 新定义计数器
\newcounter{FA}[chapter]
\renewcommand{\theFA}{\thechapter.\arabic{FA}}

%新定义定理环境
\theoremstyle{plain}
%\theoremstyle{Thm}
\newtheorem{theorem}[FA]{Theorem}
\newtheorem*{theoremn}{Theorem}
\newtheorem{lemma}[FA]{Lemma}
\newtheorem{remark}[FA]{Remark}
\newtheorem*{remarkn}{Remark}
\newtheorem*{solution}{Solution}
\newtheorem{exercise}{Exercise}
\newtheorem{problem}{Problem}
\newtheorem{definition}[FA]{Definition}
\newtheorem*{definitionn}{Definition}
\newtheorem{corollary}[FA]{Corollary}
\newtheorem{example}[FA]{Example}
\newtheorem{proposition}[FA]{Proposition}
\newtheorem{extraexample}{Extraexample}
\newtheorem*{lemman}{Lemma} %
\newtheorem*{propositionn}{Proposition}
\newtheorem*{corollaryn}{Corollary} 

\newtheoremstyle{plain}{\topsep}{\topsep}{\normalfont}{}{%
	\color{blue}\bfseries}{}{%
	0.5em}{%
	\thmname{#1}\thmnumber{ #2}\thmnote{ (#3)}}

\renewcommand*{\proofname}{%
	\normalfont\bfseries\color{blue} Proof}

%新定义命令
\newcommand{\abs}[1]{\ensuremath{\left| #1 \right| }}
\newcommand{\norm}[1]{\ensuremath{\left\| #1 \right\|}}
\newcommand{\tabs}[1]{\ensuremath{\lvert #1\rvert}}
\newcommand{\tnorm}[1]{\ensuremath{\lVert #1\rVert}}
\newcommand{\Babs}[1]{\ensuremath{\Big| #1 \Big| }}
\newcommand{\Bnorm}[1]{\ensuremath{\Big\| #1 \Big\|}}
\newcommand{\lrangle}[1]{\left\langle #1 \right\rangle}
\newcommand{\degree}{\ensuremath{^{\circ}}}
\newcommand{\sm}{\ensuremath{\setminus}}
\newcommand{\baro}[1]{\overline{#1}}
\newcommand{\weakto}{\ensuremath{\overset{w.}{\longrightarrow}}}
\newcommand{\sweakto}{\ensuremath{\overset{\Star{w.}}{\longrightarrow}}}
\newcommand{\seq}[2][n]{\ensuremath{#2_{1}, #2_{2}, \dots, #2_{#1}}}
\renewcommand{\thefootnote}{$\sharp$\arabic{footnote}}
%\renewcommand{\labelenumi}{(\theenumi)}


\usepackage{float}
\usepackage{pdfpages}

\makeindex[intoc]
\renewcommand{\qedsymbol}{$\blacksquare$}
\pagestyle{plain}

\usepackage{pgfplots}
% 图画
\addtolength{\jot}{0.5em}

% 如果图片没有指定后缀, 依次按下列顺序搜索
\DeclareGraphicsExtensions{.pdf,.eps,.jpg,.png}
% 设置图表搜索路径, 可以给图表文件夹取如下名字
\graphicspath{{figures/}{figure/}{pictures/}%
	{picture/}{pic/}{pics/}{image/}{images/}}

\renewcommand{\contentsname}{Content}
\usepackage{algorithm}
\usepackage{algpseudocode}

% INPUT和OUTPUT
\renewcommand{\algorithmicrequire}{\textbf{Input:}}
\renewcommand{\algorithmicensure}{\textbf{Output:}}

\renewcommand{\bibname}{References}
% 修改参考文献标题



\usepackage{tcolorbox}

\tcbuselibrary{breakable}
\newtcolorbox{pbox}[1][]{
	colframe=gray!85, % 边框颜色
	colback=white,       % 背景色
	title={\textbf{#1}}, % 标题文本
	% before upper={\setlength{\parindent}{1.5em}}, % 设置首行缩进
	% before lower={\setlength{\parindent}{1.5em}}, % 下面首行不缩进
	breakable,           % 允许跨页
}


\usepackage{svg}
