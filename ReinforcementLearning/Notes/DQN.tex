\section{DQN}

\subsection{DQN Method}

DQN(Deep Q-Network)算法最终更新的目标是让动作价值函数 $Q_\omega(s,a)$ 逼近时序差分(TD)目标 $r + \gamma \max_{a'} Q_\omega(s',a')$,其中 $\omega$ 表示网络参数,$s$ 为当前状态,$a$ 为当前动作,$r$ 为即时回报,$s'$ 为下一状态,$\gamma \in [0,1]$ 为折扣因子。

由于 TD 误差目标本身包含神经网络的输出,因此在更新网络参数 $\omega$ 的同时,目标值也会随之不断改变,这容易导致神经网络训练的不稳定性(如参数震荡、收敛缓慢)。

为解决这一问题,DQN 引入了{\color{red} 目标网络(Target Network) }的核心思想:暂时固定 TD 目标中的 $Q$ 网络,避免目标值随训练网络实时波动。


\begin{tcolorbox}[title={具体实现}]
需构建两套结构完全相同但参数更新不同步的 $Q$ 网络
\tcblower

\begin{enumerate}
    \item 训练网络 $Q_\omega(s,a)$:用于计算损失函数中的预测项 $Q_\omega(s,a)$,通过梯度下降实时更新参数 $\omega$;
    \item 目标网络 $Q_{\omega^-}(s,a)$:用于计算 TD 目标项 $r + \gamma \max_{a'} Q_{\omega^-}(s',a')$,其中 $\omega^-$ 为目标网络参数。目标网络不随训练网络实时更新,而是每隔 $C$ 步(超参数)复制一次训练网络的当前参数,即 $\omega^- \leftarrow \omega$,以此保证目标值的稳定性。
\end{enumerate}
\end{tcolorbox}


\noindent 此时,DQN 的损失函数修正为:
\[
\boxed{ L(\omega) = \frac{1}{N} \sum_{i=1}^N \left[ Q_\omega(s_i,a_i) - \left( r_i + \gamma \max_{a'} Q_{\omega^-}(s_{i+1},a') \right) \right]^2 }
\]
其中 $N$ 为经验回放池的采样批量大小。

\subsection{DQN Algorithm}

\begin{algorithm}[H]  % H:强制当前位置排版(可改为t/b/p自动排版)
\caption{DQN 算法完整流程}
\label{alg:dqn_full}  % 算法标签,用于交叉引用
\begin{algorithmic}[1]  % [1]:显示行号
    \REQUIRE 环境状态空间 $\mathcal{S}$、动作空间 $\mathcal{A}$、折扣因子 $\gamma$、学习率 $\alpha$、
             探索率 $\epsilon$(衰减策略)、目标网络更新间隔 $C$、经验回放池容量 $M$、采样批量 $N$、
             训练总序列数 $E$、每个序列最大时间步 $T$
    \ENSURE 训练稳定的动作价值网络 $Q_\omega(s,a)$
    
    % 初始化步骤
    \STATE 随机初始化训练网络参数 $\omega$,构建 $Q_\omega(s,a)$
    \STATE 复制参数 $\omega^- \leftarrow \omega$,初始化目标网络 $Q_{\omega^-}(s,a)$
    \STATE 初始化经验回放池 $R$(容量为 $M$)
    \STATE 初始化探索率 $\epsilon$(如 $\epsilon_{\text{init}} = 1.0$)
    
    % 序列循环(Episode Loop)
    \FOR{$e = 1$ \TO $E$}
        \STATE 获取环境初始状态 $s_1$(将状态转换为网络输入格式,如向量/矩阵)
        \STATE 重置当前序列的累计回报 $R_{\text{total}} = 0$
        
        % 时间步循环(Time Step Loop)
        \FOR{$t = 1$ \TO $T$}
            % 1. 基于ε-贪婪策略选择动作
            \STATE 生成随机数 $rand \sim U(0,1)$
            \IF{$rand < \epsilon$}
                \STATE 随机选择动作 $a_t \in \mathcal{A}$(探索)
            \ELSE
                \STATE 选择贪婪动作 $a_t = \arg\max_{a \in \mathcal{A}} Q_\omega(s_t, a)$(利用)
            \ENDIF
            
            % 2. 与环境交互
            \STATE 执行动作 $a_t$,获得即时回报 $r_t$、下一状态 $s_{t+1}$、终止标志 $done$
            \STATE $R_{\text{total}} = R_{\text{total}} + r_t$
            
            % 3. 存储经验到回放池
            \STATE 将经验元组 $(s_t, a_t, r_t, s_{t+1}, done)$ 存入 $R$(若容量超 $M$,替换旧经验)
            
            % 4. 经验回放更新训练网络
            \IF{$\text{len}(R) \geq N$}  % 回放池数据量足够采样
                \STATE 从 $R$ 中随机采样 $N$ 个经验 $\{(s_i, a_i, r_i, s_{i+1}, done_i)\}_{i=1}^N$
                \STATE 对每个采样经验计算 TD 目标 $y_i$:
                \[
                y_i = 
                \begin{cases} 
                r_i & \text{若 } done_i = \text{True(终止状态)} \\
                r_i + \gamma \max_{a' \in \mathcal{A}} Q_{\omega^-}(s_{i+1}, a') & \text{否则}
                \end{cases}
                \]
                \STATE 计算损失函数 $L = \frac{1}{N} \sum_{i=1}^N (y_i - Q_\omega(s_i, a_i))^2$
                \STATE 基于梯度下降更新训练网络参数 $\omega$($\omega \leftarrow \omega - \alpha \nabla_\omega L$)
            \ENDIF
            
            % 5. 更新目标网络(每隔C步)
            \IF{$t \mod C == 0$}
                \STATE 复制训练网络参数到目标网络:$\omega^- \leftarrow \omega$
            \ENDIF
            
            % 6. 终止状态判断
            \IF{$done$}
                \STATE 输出当前序列回报 $R_{\text{total}}$,跳出时间步循环
            \ELSE
                \STATE $s_t = s_{t+1}$(状态转移)
            \ENDIF
        \ENDFOR
        
        % 7. 探索率衰减(降低探索比例)
        \STATE $\epsilon = \epsilon \times \epsilon_{\text{decay}}$(如 $\epsilon_{\text{decay}} = 0.995$)
        \STATE $\epsilon = \max(\epsilon_{\text{min}}, \epsilon)$(设置最小探索率,如 $\epsilon_{\text{min}} = 0.01$)
    \ENDFOR
\end{algorithmic}
\end{algorithm}
\begin{itemize}
\item 经验回放池:避免样本相关性导致的训练不稳定,通过随机采样打破经验的时序关联;
\item 目标网络:固定 TD 目标值,减少参数更新时的目标波动,提升收敛稳定性;
\item $\epsilon$-贪婪策略:平衡探索(发现新动作)与利用(选择已知最优动作),通过衰减 $\epsilon$ 逐步降低探索比例。
\end{itemize}

