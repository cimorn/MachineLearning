\section{Double DQN}

DQN算法通过经验回放和目标网络两大机制,解决了高维状态空间下Q学习的训练不稳定性问题,但仍存在一个关键缺陷——Q值过估计(Overestimation Bias)。

这种过估计源于DQN的目标值计算方式:使用同一目标网络同时完成“动作选择”和“价值评估”,神经网络的逼近误差会被$\max$操作放大,导致Q值系统性偏高,进而引发策略次优、收敛速度减慢等问题。

为解决这一问题,便有了Double DQN(DDQN) 算法,其核心思想源于经典的Double Q-Learning:将动作选择与价值评估解耦,通过两个独立的网络分别完成这两项任务,从而打破过估计的正向反馈循环,显著提升Q值估计的准确性和算法稳定性。

“解耦”(Decoupling) 指的是将 DQN 中 “动作选择” 和 “价值评估” 这两个原本绑定在同一网络的任务,拆分给两个独立网络分别完成,避免单一网络的偏差被放大,核心是 “职责分离、相互独立”。

\subsection{DDQN and DQN}
\begin{table}[H]
\centering
\renewcommand{\arraystretch}{1.8}  % 增加表格行间距
\begin{tabular}{|l|l|l|}
\hline
\textbf{对比维度}       & \textbf{DQN算法}                          & \textbf{DDQN算法}                          \\
\hline
目标值计算逻辑 & 目标网络同时负责动作选择与评估    & 在线网络选动作,目标网络做评估    \\
\hline
目标值公式     & $y_t^{DQN} = r_t + \gamma \max_{a'} Q_{\theta^-}(s_{t+1}, a')$ & 
\begin{tabular}[c]{@{}l@{}}
$y_t^{DDQN} = r_t + \gamma Q_{\theta^-}\big(s_{t+1},$ \\
$~~~~~~~~~~~~~~~\arg\max_{a'} Q_{\theta}(s_{t+1}, a')\big)$
\end{tabular} \\
\hline
过估计风险     & 高(单一网络偏差被$\max$放大)    & 低(双网络解耦,偏差相互抵消)    \\
\hline
网络结构       & 双网络(训练网络$+$目标网络)       & 双网络(在线网络$+$目标网络,结构一致) \\
\hline
实现成本       & 基础版本                         & 仅修改目标值计算,无额外开销      \\
\hline
\end{tabular}
\caption{DQN and DDQN}
\label{tab:dqn_ddqn_compare}
\end{table}

% 核心机制框选突出
\begin{tcolorbox}[title={DDQN核心机制:解耦动作选择与价值评估}]
DDQN沿用DQN的“双网络”架构,但赋予两个网络明确的分工:

1. 在线网络($Q_\theta(s,a)$):负责动作选择,即通过$\arg\max_{a'} Q_\theta(s_{t+1}, a')$确定下一状态的最优动作,反映当前训练策略的偏好;

2. 目标网络($Q_{\theta^-}(s,a)$):负责价值评估,即对在线网络选择的最优动作进行Q值计算,提供稳定的目标值基准。
\end{tcolorbox}

这种分工的关键优势在于:即使两个网络都存在估计误差,它们同时高估同一动作价值的概率极低,从而有效抑制过估计偏差的累积。

\subsection{DDQN Method}
DDQN的损失函数形式与DQN一致,仍采用均方误差(MSE)损失,但目标值替换为解耦计算后的$y_t^{DDQN}$:
\[
\boxed{L(\theta) = \frac{1}{N} \sum_{i=1}^N \left[ Q_\theta(s_i,a_i) - y_i^{DDQN} \right]^2}
\]
其中,$y_i^{DDQN}$为DDQN的目标值,分两种情况计算:
\[
y_i^{DDQN} = 
\begin{cases} 
r_i & \text{若 } done_i = \text{True(终止状态,无未来奖励)} \\
r_i + \gamma Q_{\theta^-}(s_{i+1}, \arg\max_{a'} Q_\theta(s_{i+1}, a')) & \text{否则}
\end{cases}
\]
$\gamma \in [0,1]$为折扣因子,$N$为经验回放池的采样批量大小。

\vspace{2em}
\noindent \textbf{DDQN算法优势}:
\begin{itemize}
\item 显著缓解过估计:实验证明在Atari游戏等任务中,DDQN的平均Q值更接近真实值,策略更稳健;
 \item 兼容性强:可与DQN的其他改进(如优先经验回放、Dueling架构)无缝结合;
 \item 稳定性提升:目标值计算更可靠,避免因过估计导致的策略震荡,收敛速度更快。
\end{itemize}

\subsection{DDQN Algorithm}

\begin{algorithm}[H]
\caption{Double DQN(DDQN)完整算法流程}
\label{alg:ddqn_full}
\begin{algorithmic}[1]  % 显示行号
    \REQUIRE 环境状态空间 $\mathcal{S}$、动作空间 $\mathcal{A}$、折扣因子 $\gamma$、学习率 $\alpha$、
             探索率 $\epsilon$(衰减策略)、目标网络更新间隔 $C$、经验回放池容量 $M$、采样批量 $N$、
             训练总序列数 $E$、每个序列最大时间步 $T$
    \ENSURE 低偏差的动作价值网络 $Q_\theta(s,a)$
    
    % 初始化步骤
    \STATE 随机初始化在线网络参数 $\theta$,构建 $Q_\theta(s,a)$(负责动作选择与预测)
    \STATE 复制参数 $\theta^- \leftarrow \theta$,初始化目标网络 $Q_{\theta^-}(s,a)$(负责价值评估)
    \STATE 初始化经验回放池 $R$(容量为 $M$)
    \STATE 初始化探索率 $\epsilon$(如 $\epsilon_{\text{init}} = 1.0$,最小探索率 $\epsilon_{\text{min}} = 0.01$)
    
    % 序列循环(Episode Loop)
    \FOR{$e = 1$ \TO $E$}
        \STATE 获取环境初始状态 $s_1$(转换为网络输入格式,如向量/矩阵)
        \STATE 重置当前序列累计回报 $R_{\text{total}} = 0$
        
        % 时间步循环(Time Step Loop)
        \FOR{$t = 1$ \TO $T$}
            % 1. 基于ε-贪婪策略选择动作(在线网络决策)
            \STATE 生成随机数 $rand \sim U(0,1)$
            \IF{$rand < \epsilon$}
                \STATE 随机选择动作 $a_t \in \mathcal{A}$(探索)
            \ELSE
                \STATE 在线网络选贪婪动作:$a_t = \arg\max_{a \in \mathcal{A}} Q_\theta(s_t, a)$(利用)
            \ENDIF
            
            % 2. 与环境交互
            \STATE 执行动作 $a_t$,获得即时回报 $r_t$、下一状态 $s_{t+1}$、终止标志 $done$
            \STATE $R_{\text{total}} = R_{\text{total}} + r_t$
            
            % 3. 存储经验到回放池
            \STATE 将经验元组 $(s_t, a_t, r_t, s_{t+1}, done)$ 存入 $R$(超容量时替换旧经验)
            
            % 4. 经验回放更新在线网络(核心步骤:解耦目标值计算)
            \IF{$\text{len}(R) \geq N$}  % 回放池数据量满足采样要求
                \STATE 从 $R$ 中随机采样 $N$ 个经验 $\{(s_i, a_i, r_i, s_{i+1}, done_i)\}_{i=1}^N$
                \STATE 对每个采样经验,用在线网络选择下一状态最优动作:
                    $a_i^* = \arg\max_{a' \in \mathcal{A}} Q_\theta(s_{i+1}, a')$
                \STATE 用目标网络计算DDQN目标值 $y_i^{DDQN}$:
                \[
                y_i^{DDQN} = 
                \begin{cases} 
                r_i & \text{若 } done_i = \text{True} \\
                r_i + \gamma Q_{\theta^-}(s_{i+1}, a_i^*) & \text{否则}
                \end{cases}
                \]
                \STATE 计算损失函数 $L = \frac{1}{N} \sum_{i=1}^N (y_i^{DDQN} - Q_\theta(s_i, a_i))^2$
                \STATE 基于梯度下降更新在线网络参数:$\theta \leftarrow \theta - \alpha \nabla_\theta L$
            \ENDIF
            
            % 5. 定期更新目标网络(与DQN一致,保证稳定性)
            \IF{$t \mod C == 0$}
                \STATE 复制在线网络参数到目标网络:$\theta^- \leftarrow \theta$
            \ENDIF
            
            % 6. 终止状态判断
            \IF{$done$}
                \STATE 输出当前序列回报 $R_{\text{total}}$,跳出时间步循环
            \ELSE
                \STATE $s_t = s_{t+1}$(状态转移)
            \ENDIF
        \ENDFOR
        
        % 7. 探索率衰减(平衡探索与利用)
        \STATE $\epsilon = \epsilon \times \epsilon_{\text{decay}}$(如 $\epsilon_{\text{decay}} = 0.995$)
        \STATE $\epsilon = \max(\epsilon_{\text{min}}, \epsilon)$(限制最小探索率)
    \ENDFOR
\end{algorithmic}
\end{algorithm}

\textbf{算法说明}
\begin{itemize}
\item 目标网络更新:仍采用“硬更新”策略(每隔$C$步复制参数),也可采用“软更新”($\theta^- \leftarrow \tau\theta + (1-\tau)\theta^-$,$\tau$为小权重如0.001),进一步提升稳定性;
\item 经验回放池:功能与DQN一致,用于打破样本时序相关性,避免参数更新偏向近期经验;
\item 适用场景:适用于所有DQN可解决的任务,尤其在奖励稀疏、动作空间复杂的环境中,过估计缓解效果更显著(如机器人导航、复杂游戏AI);
\item 扩展方向:DDQN是Rainbow DQN(集大成改进算法)的核心组件之一,可与优先经验回放(PER)、竞争网络(Dueling DQN)等结合,进一步提升性能。
\end{itemize}

